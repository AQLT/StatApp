\tikzstyle{myboxnorm} = [very thick,
    rectangle, rounded corners, inner sep=2pt, inner ysep=3pt, right]
\tikzstyle{myboxw2v} = [draw=red!50, fill=orange!10, myboxnorm]
\tikzstyle{myboxmod} = [draw=blue!60!green, fill=blue!5, myboxnorm]
\tikzstyle{myboxcomp} = [draw=green!50!black, fill=green!5, myboxnorm]

\tikzstyle{titlenorm} = [fill=white, very thick,
    rectangle, inner sep=2pt, inner ysep=2pt,font=\bfseries,
    text width=4.2cm, above=-0.3cm, align=center,
    minimum height=1.15cm]
\tikzstyle{titlew2v} = [draw=red!50, text = red!50,titlenorm]
\tikzstyle{titlemod} = [draw=blue!60!green,text=blue!60!green,titlenorm]
\tikzstyle{titlecomp} = [draw=green!50!black,text=green!50!black, titlenorm]    

\tikzstyle{fleche} = [->,rounded corners,line width=1pt]

\begin{tikzpicture}
\usetikzlibrary{fit}
\usetikzlibrary{arrows.meta}
%Il faut d'abord faire les boites
% w2vec
\node[fit={(2.5,1.5) (-3.,-3.5)}, myboxw2v,semitransparent] (w2vrect) {};
\node[fit={(2.7,1.5) (8.2,-5.5)}, myboxmod,semitransparent] (rectmod) {};
\node[fit={(13.9,1.5) (8.4,-5.5) }, myboxcomp,semitransparent] (rectcomp) {};

\node at (0,0) [myboxw2v] (tweets) {Tweets};
\node at (2.4,0) [myboxw2v] (we) {\textit{word-embedding}};
% sentiment
\node at (0,-4.5) [myboxnorm,text width=2.5cm, draw = black] (bddsent) {Échantillon de tweets annotés
};

% analyse de sentimentsentiment
\node at (7,0) [myboxmod] (logit) {Modèle logit};
\node at (7,-4.5) [myboxmod,text width=2cm] (baseline) {Modèle de référence};

% analyse de sentimentsentiment
\node at (12,0) [myboxcomp,text width=4cm] (indsent) {Indices mensuels de sentiment des tweets};

\node at (12,-4.5) [myboxcomp,text width=4cm] (camme) {Indicateur synthétique de confiance des

ménages (Insee)};

%Fleches
\draw[fleche] (tweets.east) --(we.west) node[below = 0.2cm,pos=1]{
\begin{minipage}{5cm} \footnotesize
  \begin{itemize}[label=\scalebox{.6}{\ding{110}}] 
  \item tokénisation
  \item choix hyperparamètres
  \item évaluation : 
  \begin{itemize}[label=\scalebox{.6}{\ding{117}}]
    \item similarité cosinus
    \item réduction de dimension (ACP/TSNE)
    \item jugement humain
  \end{itemize}
  \end{itemize} 
\end{minipage} 
};
\draw[fleche] (bddsent.east)--(baseline.west) 
  node[pos=0.36,text width=3cm]{
    \footnotesize bases  
    
    test/entraînement}
  node[pos=0.96, below = 0.05cm,text width=2.5cm]{
    \footnotesize sentiment 
    
    moyen d'un mot};
\draw[fleche] (bddsent.east)--++(2,0)--++(0,0.8)-|(logit.south);
\draw[fleche] (we.east)--(logit.west);
\draw[fleche] (logit.east)--
  (indsent.west)node[pos=0.5,text width=2.5cm]{\footnotesize moyennes
  
  mensuelles};
\draw[fleche] (baseline.east)-|(11,0)--(indsent.west);
\draw[fleche,<->] (indsent.south)--(camme.north) 
  node[pos=.5,text width=2.5cm, right]{\footnotesize distance entre
  
  indicateurs}
  node[pos=.5,text width=1.7cm, left]{\footnotesize prévision /
  
  causalité};

\node[titlew2v] at (w2vrect.north) { word2vec \\
(skip-gram)};
\node[titlemod] at (rectmod.north) { Analyse de \\
sentiment d'un tweet};
\node[titlecomp] at (rectcomp.north) { Comparaison \\
d'indices mensuels};
\end{tikzpicture}