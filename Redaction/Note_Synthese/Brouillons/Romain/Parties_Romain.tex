\documentclass[11pt,french,french]{article}
\usepackage{lmodern}
\usepackage{amssymb,amsmath}
\usepackage{ifxetex,ifluatex}
\usepackage{fixltx2e} % provides \textsubscript
\ifnum 0\ifxetex 1\fi\ifluatex 1\fi=0 % if pdftex
  \usepackage[T1]{fontenc}
  \usepackage[utf8]{inputenc}
\else % if luatex or xelatex
  \ifxetex
    \usepackage{mathspec}
    \usepackage{xltxtra,xunicode}
  \else
    \usepackage{fontspec}
  \fi
  \defaultfontfeatures{Mapping=tex-text,Scale=MatchLowercase}
  \newcommand{\euro}{€}
\fi
% use upquote if available, for straight quotes in verbatim environments
\IfFileExists{upquote.sty}{\usepackage{upquote}}{}
% use microtype if available
\IfFileExists{microtype.sty}{%
\usepackage{microtype}
\UseMicrotypeSet[protrusion]{basicmath} % disable protrusion for tt fonts
}{}
\usepackage[margin=0.95in]{geometry}
\ifxetex
  \usepackage{polyglossia}
  \setmainlanguage{}
\else
  \usepackage[shorthands=off,french]{babel}
\fi
\ifxetex
  \usepackage[setpagesize=false, % page size defined by xetex
              unicode=false, % unicode breaks when used with xetex
              xetex]{hyperref}
\else
  \usepackage[unicode=true]{hyperref}
\fi
\hypersetup{breaklinks=true,
            bookmarks=true,
            pdfauthor={},
            pdftitle={},
            colorlinks=true,
            citecolor=blue,
            urlcolor=blue,
            linkcolor=magenta,
            pdfborder={0 0 0}}
\urlstyle{same}  % don't use monospace font for urls
\setlength{\parindent}{0pt}
\setlength{\parskip}{6pt plus 2pt minus 1pt}
\setlength{\emergencystretch}{3em}  % prevent overfull lines
\setcounter{secnumdepth}{5}

%%% Use protect on footnotes to avoid problems with footnotes in titles
\let\rmarkdownfootnote\footnote%
\def\footnote{\protect\rmarkdownfootnote}


  \title{Analyse statistique et empirique des modèles\\
de Word Embedding sur Twitter}
    \author{Kim Antunez, Romain Lesauvage, Alain Quartier-la-Tente\\
sous l'encadrement de Benjamin Muller (Inria)}
    \date{}
  
\usepackage{caption}
\usepackage{graphicx}
\usepackage{natbib}

\begin{document}

\maketitle


\section{Évaluation du modèle
implémenté}\label{uxe9valuation-du-moduxe8le-impluxe9mentuxe9}

\subsection{Comment évaluer évaluer le modèle
?}\label{comment-uxe9valuer-uxe9valuer-le-moduxe8le}

\subsubsection{Similarité cosinus}\label{similarituxe9-cosinus}

Afin de pouvoir évaluer le modèle, nous devons fixer un certain nombre
critères sur lesquels s'appuyer. L'un des enjeux principaux est de
pouvoir estimer la proximité entre deux mots. En effet, le modèle doit
capter grâce aux vecteurs ces notions de proximité entre mots. Il faut
donc se fixer une distance afin de mesurer ce phénomène.

Il existe différents types de distances que nous pouvons classiquement
utiliser pour mesurer la proximité entre deux vecteurs, parmi elles :

\begin{itemize}
\item la distance euclidienne $ d(\vec{u},\vec{v}) = \left\| \vec{u} - \vec{v}  \right\|_2$ ;
\item la distance de Manhattan $ d(\vec{u},\vec{v}) = |x_{\vec{u}} - x_{\vec{v}} | + |y_{\vec{u}} - y_{\vec{v}} |$ ;
\item la similarité cosinus $ d(\vec{u}, \vec{v}) = \frac{\vec{u}.\vec{v}}{\left\| \vec{u} \right\|_2  \left\| \vec{v} \right\|_2 }$ .
\end{itemize}

Chacune de ces distances possèdent des propriétés intéressantes mais
l'utilisation d'une d'entre elles en particulier dépend du problème sur
lequel on est en train de travailler. Dans le cadre de \emph{Word2Vec},
nous préférerons ici la similarité cosinus car elle s'intéresse plutôt à
l'angle formé entre deux vecteurs, tandis que les distances euclidiennes
et de Manhattan s'intéressent aux valeurs du vecteur directement, or
l'angle est un paramètre plus robuste dans notre cas.

Nous pouvons donc interpréter ici la similarité cosinus comme une mesure
de l'angle formé entre deux vecteurs représentant deux mots différents.
Aussi, une similarité proche de +1 signifiera que les mots sont très
proches, une valeur proche de -1 que les mots sont corrélés négativement
et une valeur proche de 0 une quasi-indépendance.

\subsubsection{Analyse en Composantes
Principales}\label{analyse-en-composantes-principales}

Une fois que le modèle \emph{Word2Vec} est entraîné, nous obtenons des
\emph{word-embeddings} pour chacun de nos mots, représentés par des
vecteurs de dimension pouvant être 20, 50 ou même 100. Dès lors, il est
très compliqué d'avoir un outil de visualisation permettant de repérer
les mots qui semblent proches. Pour parer à cela, nous pouvons alors
utiliser une analyse en composantes principales. L'objectif premier de
cette analyse est d'arriver à projeter un nuage de points sur un espace
de dimension inférieure, tout en étant le plus proche de la réalité.

Considérons le cas où nous disposons de \(n\) individus et de \(p\)
variables. On note \(X = (x_{ij})\) la matrice de taille \((n,p)\) des
données brutes, où \(x_{ij}\) représente la valeur de la \(j\)-ème
variable pour le \(i\)-ème individu. On munit notre espace de la
distance euclienne, on peut alors mesurer la distance entre deux
individus \(x_i\) et \(x_{i'}\) en utilisant :
\[d(x_i,x_{i'}) = \sum \limits_{j=1}^p (x_{ij} - x_{i'j})^2 \]

On peut commencer par remarquer que si l'on centre et/ou réduit notre
matrice de données, la distance en deux éléments restera inchangée mais
que cela permet de donner le même poids à tous les individus. On notera
dans la suite \(Y = (y_{ij})\) la matrice des données centrées et
\(Z = (z_{ij})\) celle des données centrées et réduites.

La dispersion du nuage de points se mesure grâce à l'inertie, que l'on
définit comme
\[I(X) = \frac{1}{n} \sum \limits_{i = 1}^n d^2(x_i,\bar{x})\]. C'est
une généralisation de la notion de variance dans le cadre multivarié et
c'est cette quantité que l'ACP cherche à représenter au mieux sur les
axes factoriels.

Considérons désormais la métrique
\(N = diag(\frac{1}{n},...,\frac{1}{n})\), on peut alors définir le
produit scalaire \(<x,y> = x\,^t N y\). En notant \(x^j\) la vecteur
pour la \(j\)-ème variable, on a \(var(x^j) = \left\| y^j \right\|^2\)
et de même \(var(z^j) = \left\| z^j \right\|^2\).

Pour trouver les axes de l'ACP, on va désormais procéder par projection
orthogonale. Rappelons que les coordonnées de projection des \(n\)
individus sur une droite de vecteur directeur \(v_\alpha\) est
\(f^\alpha = Zv_\alpha\). Nous allons donc chercher un vecteur \(v_1\)
vecteur directeur du premier axe tel que
\[v_1 =\underset{ \left\| v  \right\| = 1}{\mathrm{argmin~}} Var(Zv) =\underset{ \left\| v  \right\| = 1}{\mathrm{argmin~}} v\,^t R v \]
où \(R = \frac{1}{n} Z\,^t Z\) est la matrice des corrélations entre les
p variables. On peut alors montrer que \(v_1\) est un vecteur propre
associé à la première valeur propre \(\lambda_1\) de \(R\). Il reste à
trouver \(v_2\) orthogonal à \(v_1\) tel quel l'intertie soit maximisée
:
\[v_2 =\underset{ \left\| v  \right\| = 1, v \perp v_1}{\mathrm{argmin}} Var(Zv) \]
On obtient alors que \(v_2\) est un vecteur propre associé à la deuxième
valeur propre \(\lambda_2\) de \(R\) et, en procédant de manière
séquentielle, on obtient \(q < r\) axes orthogonaux sur lesquels on
effectue notre projection, où \(r\) est le rang de la matrice \(Z\).

\subsubsection{Algorithme t-distributed Stochastic Neighbor
Embedding}\label{algorithme-t-distributed-stochastic-neighbor-embedding}

Bien que l'ACP soit une première manière de résumer l'information
contenue dans nos vecteurs, elle présente des limites, notamment dans
les vecteurs aux trop grandes dimensions, pour lesquels l'inertie des
premiers axes de l'ACP peut se révéler faible. Pour combler ces lacunes,
un autre algorithme peut être utilisé, celui dit du t-distributed
Stochastic Neighbor Embedding. Contraitement à l'ACP, cet algorithme est
stochastique et favorise l'apparition de groupes de mots proches. L'idée
reste cependant la même, c'est-à-dire pouvoir représenter dans un espace
l'ensemble de notre nuage de points de manière à repérer les mots
proches.

L'algorithme commence par transformer les distances euclidiennes entre
nos vecteurs \((X_1,...,X_n)\) en probabilités conditionnelles qui
représenteront également les similarités entre nos vecteurs. Pour cela,
on note :

\[ p_{j|i} = \frac{\exp{-\frac{\left\| \vec{X_i} - \vec{X_j}  \right\|^2}{2\sigma_i^2}}}{\sum_{k \neq i}{\exp{-\frac{\left\| \vec{X_i} - \vec{X_k}  \right\|^2}{2\sigma_i^2}}}}\]

La similarité entre les points \(X_i\) et \(X_j\) est alors la
probabilité conditionnelle que \(X_i\) choisirait \(X_j\) comme voisin
si ces derniers étaient choisis selon une loi gaussienne centrée en
\(X_i\). Ici, \(\sigma_i\) est la variance de cette gaussienne centrée
en \(X_i\), calculer de manière itérative de sorte à maximiser la
perplexité du modèle (mesure du modèle de probabilités).

Dans une seconde étape, il faut réaliser les mêmes calculs mais dans
l'espace de projection, où nous aurons nos points \((Y_1,..., Y_n)\).
Cependant, puisqu'en dimension réduite, la gaussienne à tendance à
concentrer les points, l'algorithme propose d'utilise une distribution
de \emph{Student}, d'où le nom de l'algorithme, et ainsi de calculer :

\[ q_{j|i} = \frac{(1+\left\| \vec{Y_i} - \vec{Y_j}  \right\|^2)^{-1}}{\sum_{k \neq i}{(1+\left\| \vec{Y_i} - \vec{Y_k}  \right\|^2)^{-1}}}\]

Il ne reste plus désormais qu'à déterminer les vecteurs
\((Y_1,...,Y_n)\) qui minimise l'écart entre les deux distributions de
probabilité. Pour cela, l'algorithme se base sur la divergence de
Kullback--Leibler entre les distributions P et Q :

\[KL(P,Q) = \sum_{i \neq j} { p_{ij} \log{\frac{p_{ij}}{q_{ij}}}}\]

Avec \[p_{ij} = \frac{p_{i|j} + p_{j|i}}{2n}\] où \(n\) est le nombre de
vecteurs.

Ensuite, on minimise grâce à une descente de gradient. On obtient alors
les résultats de l'algorithme t-SNE, qu'il faut cependant analyser avec
précaution : il ne s'agît pas d'un algorithme linéaire équivalent à
l'ACP, on ne peut donc pas interpréter directement des éléments tels que
la taille des clusters obtenus ou leur distance relative.

\section{Pourquoi on utilise la cosinus distance et pas la distance
euclienne}\label{pourquoi-on-utilise-la-cosinus-distance-et-pas-la-distance-euclienne}

\url{https://www.quora.com/Why-do-we-use-cosine-similarity-on-Word2Vec-instead-of-Euclidean-distance}
Cosine distance + robuste ?

Measuring Word Significance using Distributed Representations of Words :
\url{https://arxiv.org/pdf/1508.02297v1.pdf}

\begin{itemize}
\item La magnitude/longueur du vecteur est liée à la fréquence d'apparition des mots donc attention à l'usage de la distance euclidienne.
\item During training, words that appear in similar contexts are grouped together in the **same direction** by this unsupervised learning algorithm. 
The distributed representation thus ultimately captures semantic similarities between words. 
This has been impressively demonstrated by a series of experiments in the original word2vec papers, where semantic similarity was measured by the dot product between normalized vectors.
\end{itemize}

\url{https://arxiv.org/pdf/1512.00765.pdf}

\end{document}