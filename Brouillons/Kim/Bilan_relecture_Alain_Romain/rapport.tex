\documentclass[11pt,french,french]{article}
\usepackage{lmodern}
\usepackage{amssymb,amsmath}
\usepackage{ifxetex,ifluatex}
\usepackage{fixltx2e} % provides \textsubscript
\ifnum 0\ifxetex 1\fi\ifluatex 1\fi=0 % if pdftex
  \usepackage[T1]{fontenc}
  \usepackage[utf8]{inputenc}
\else % if luatex or xelatex
  \ifxetex
    \usepackage{mathspec}
    \usepackage{xltxtra,xunicode}
  \else
    \usepackage{fontspec}
  \fi
  \defaultfontfeatures{Mapping=tex-text,Scale=MatchLowercase}
  \newcommand{\euro}{€}
\fi
% use upquote if available, for straight quotes in verbatim environments
\IfFileExists{upquote.sty}{\usepackage{upquote}}{}
% use microtype if available
\IfFileExists{microtype.sty}{%
\usepackage{microtype}
\UseMicrotypeSet[protrusion]{basicmath} % disable protrusion for tt fonts
}{}
\usepackage[margin=1in]{geometry}
\ifxetex
  \usepackage{polyglossia}
  \setmainlanguage{}
\else
  \usepackage[shorthands=off,french]{babel}
\fi
\usepackage{color}
\usepackage{fancyvrb}
\newcommand{\VerbBar}{|}
\newcommand{\VERB}{\Verb[commandchars=\\\{\}]}
\DefineVerbatimEnvironment{Highlighting}{Verbatim}{commandchars=\\\{\}}
% Add ',fontsize=\small' for more characters per line
\usepackage{framed}
\definecolor{shadecolor}{RGB}{248,248,248}
\newenvironment{Shaded}{\begin{snugshade}}{\end{snugshade}}
\newcommand{\AlertTok}[1]{\textcolor[rgb]{0.94,0.16,0.16}{#1}}
\newcommand{\AnnotationTok}[1]{\textcolor[rgb]{0.56,0.35,0.01}{\textbf{\textit{#1}}}}
\newcommand{\AttributeTok}[1]{\textcolor[rgb]{0.77,0.63,0.00}{#1}}
\newcommand{\BaseNTok}[1]{\textcolor[rgb]{0.00,0.00,0.81}{#1}}
\newcommand{\BuiltInTok}[1]{#1}
\newcommand{\CharTok}[1]{\textcolor[rgb]{0.31,0.60,0.02}{#1}}
\newcommand{\CommentTok}[1]{\textcolor[rgb]{0.56,0.35,0.01}{\textit{#1}}}
\newcommand{\CommentVarTok}[1]{\textcolor[rgb]{0.56,0.35,0.01}{\textbf{\textit{#1}}}}
\newcommand{\ConstantTok}[1]{\textcolor[rgb]{0.00,0.00,0.00}{#1}}
\newcommand{\ControlFlowTok}[1]{\textcolor[rgb]{0.13,0.29,0.53}{\textbf{#1}}}
\newcommand{\DataTypeTok}[1]{\textcolor[rgb]{0.13,0.29,0.53}{#1}}
\newcommand{\DecValTok}[1]{\textcolor[rgb]{0.00,0.00,0.81}{#1}}
\newcommand{\DocumentationTok}[1]{\textcolor[rgb]{0.56,0.35,0.01}{\textbf{\textit{#1}}}}
\newcommand{\ErrorTok}[1]{\textcolor[rgb]{0.64,0.00,0.00}{\textbf{#1}}}
\newcommand{\ExtensionTok}[1]{#1}
\newcommand{\FloatTok}[1]{\textcolor[rgb]{0.00,0.00,0.81}{#1}}
\newcommand{\FunctionTok}[1]{\textcolor[rgb]{0.00,0.00,0.00}{#1}}
\newcommand{\ImportTok}[1]{#1}
\newcommand{\InformationTok}[1]{\textcolor[rgb]{0.56,0.35,0.01}{\textbf{\textit{#1}}}}
\newcommand{\KeywordTok}[1]{\textcolor[rgb]{0.13,0.29,0.53}{\textbf{#1}}}
\newcommand{\NormalTok}[1]{#1}
\newcommand{\OperatorTok}[1]{\textcolor[rgb]{0.81,0.36,0.00}{\textbf{#1}}}
\newcommand{\OtherTok}[1]{\textcolor[rgb]{0.56,0.35,0.01}{#1}}
\newcommand{\PreprocessorTok}[1]{\textcolor[rgb]{0.56,0.35,0.01}{\textit{#1}}}
\newcommand{\RegionMarkerTok}[1]{#1}
\newcommand{\SpecialCharTok}[1]{\textcolor[rgb]{0.00,0.00,0.00}{#1}}
\newcommand{\SpecialStringTok}[1]{\textcolor[rgb]{0.31,0.60,0.02}{#1}}
\newcommand{\StringTok}[1]{\textcolor[rgb]{0.31,0.60,0.02}{#1}}
\newcommand{\VariableTok}[1]{\textcolor[rgb]{0.00,0.00,0.00}{#1}}
\newcommand{\VerbatimStringTok}[1]{\textcolor[rgb]{0.31,0.60,0.02}{#1}}
\newcommand{\WarningTok}[1]{\textcolor[rgb]{0.56,0.35,0.01}{\textbf{\textit{#1}}}}
\ifxetex
  \usepackage[setpagesize=false, % page size defined by xetex
              unicode=false, % unicode breaks when used with xetex
              xetex]{hyperref}
\else
  \usepackage[unicode=true]{hyperref}
\fi
\hypersetup{breaklinks=true,
            bookmarks=true,
            pdfauthor={Kim Antunez},
            pdftitle={Commentaires sur les travaux d'Alain et Romain},
            colorlinks=true,
            citecolor=\#FF0000,
            urlcolor=\#FF0000,
            linkcolor=\#FF0000,
            pdfborder={0 0 0}}
\urlstyle{same}  % don't use monospace font for urls
\setlength{\parindent}{0pt}
\setlength{\parskip}{6pt plus 2pt minus 1pt}
\setlength{\emergencystretch}{3em}  % prevent overfull lines
\setcounter{secnumdepth}{5}

%%% Use protect on footnotes to avoid problems with footnotes in titles
\let\rmarkdownfootnote\footnote%
\def\footnote{\protect\rmarkdownfootnote}


  \title{Commentaires sur les travaux d'Alain et Romain}
    \author{Kim Antunez}
    \date{}
  

\begin{document}

\maketitle


{
\hypersetup{linkcolor=black}
\setcounter{tocdepth}{2}
\tableofcontents
}
\hypertarget{cruxe9er-un-rapport-automatique}{%
\section{Créer un rapport
automatique}\label{cruxe9er-un-rapport-automatique}}

\hypertarget{rendu-de-rmarkdown-avec-du-code-python}{%
\subsection{Rendu de Rmarkdown avec du code
python}\label{rendu-de-rmarkdown-avec-du-code-python}}

Exemple à partir des premières lignes du code d'Alain.

\begin{Shaded}
\begin{Highlighting}[]
\ImportTok{import}\NormalTok{ os}
\ImportTok{import}\NormalTok{ string}
\ImportTok{import}\NormalTok{ re}
\ImportTok{import}\NormalTok{ math}
\ImportTok{from}\NormalTok{ math }\ImportTok{import}\NormalTok{ sqrt}
\ImportTok{import}\NormalTok{ numpy }\ImportTok{as}\NormalTok{ np}
\ImportTok{import}\NormalTok{ random}
\ImportTok{import}\NormalTok{ time}
\NormalTok{random.seed(}\DecValTok{1}\NormalTok{)}


\NormalTok{os.chdir(}\StringTok{'C:/Users/Kim Antunez/Documents/Projets_autres'}\NormalTok{)}
\BuiltInTok{print}\NormalTok{(string.punctuation }\OperatorTok{+} \StringTok{"'’"}\NormalTok{)}
\end{Highlighting}
\end{Shaded}

\begin{verbatim}
## !"#$%&'()*+,-./:;<=>?@[\]^_`{|}~'’
\end{verbatim}

\begin{Shaded}
\begin{Highlighting}[]
\KeywordTok{def}\NormalTok{ mise_en_forme_phrase (phrase):}
\NormalTok{    phrase }\OperatorTok{=}\NormalTok{ phrase.lower()}
    \CommentTok{# On elève la ponctuation mais ça peut se discuter (garder les @ et #?)}
\NormalTok{    phrase }\OperatorTok{=}\NormalTok{ re.sub(}\StringTok{'( @[^ ]*)|(^@[^ ]*)'}\NormalTok{,}\StringTok{"nickname"}\NormalTok{, phrase) }\CommentTok{#Remplace @... par nickname}
    \CommentTok{#supprime toutes les ponctuations par défaut + les apostrophes bizarres}
\NormalTok{    phrase }\OperatorTok{=}\NormalTok{ phrase.translate(}\BuiltInTok{str}\NormalTok{.maketrans(}\StringTok{''}\NormalTok{, }\StringTok{''}\NormalTok{, string.punctuation }\OperatorTok{+} \StringTok{"'’"}\NormalTok{))}
    \CommentTok{# On enlève les passages à la ligne}
\NormalTok{    phrase }\OperatorTok{=}\NormalTok{ re.sub(}\StringTok{'}\CharTok{\textbackslash{}\textbackslash{}}\StringTok{n'}\NormalTok{, }\StringTok{' '}\NormalTok{, phrase)}
    \CommentTok{# On enlève les espaces multiples et les espaces à la fin des phrases}
\NormalTok{    phrase }\OperatorTok{=}\NormalTok{ re.sub(}\StringTok{' +'}\NormalTok{, }\StringTok{' '}\NormalTok{, phrase)}
\NormalTok{    phrase }\OperatorTok{=}\NormalTok{ re.sub(}\StringTok{' +$'}\NormalTok{, }\StringTok{''}\NormalTok{, phrase)}
    \ControlFlowTok{return}\NormalTok{(phrase.split())}
\CommentTok{#f = open('data/sample_3.txt')}
\CommentTok{#raw = f.read()}
\CommentTok{#print(type(raw))}
\ControlFlowTok{with} \BuiltInTok{open}\NormalTok{(}\StringTok{'data/sample_3.txt'}\NormalTok{, encoding}\OperatorTok{=}\StringTok{"utf-8"}\NormalTok{) }\ImportTok{as}\NormalTok{ myfile:}
\NormalTok{    phrases }\OperatorTok{=}\NormalTok{ [mise_en_forme_phrase(}\BuiltInTok{next}\NormalTok{(myfile)) }\ControlFlowTok{for}\NormalTok{ x }\KeywordTok{in} \BuiltInTok{range}\NormalTok{(}\DecValTok{10000}\NormalTok{)]}
\BuiltInTok{print}\NormalTok{(phrases[}\DecValTok{0}\NormalTok{:}\DecValTok{1}\NormalTok{])}
\CommentTok{#raw = ''.join([''.join(phrase) for phrase in phrases])}
\end{Highlighting}
\end{Shaded}

\begin{verbatim}
## [['il', 'mérite', 'd', 'être', 'bloquer', 'la', 'lettre', 'de', 'l', 'alphabet']]
\end{verbatim}

\hypertarget{commentaires-des-codes-dar}{%
\section{Commentaires des codes
d'A\&R}\label{commentaires-des-codes-dar}}

\hypertarget{alain}{%
\subsection{Alain}\label{alain}}

\hypertarget{romain}{%
\subsection{Romain}\label{romain}}

\hypertarget{croisement-des-deux-codes}{%
\subsection{Croisement des deux codes}\label{croisement-des-deux-codes}}

\end{document}