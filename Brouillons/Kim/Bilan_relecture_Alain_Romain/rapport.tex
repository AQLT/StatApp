\documentclass[11pt,french,french]{article}
\usepackage{lmodern}
\usepackage{amssymb,amsmath}
\usepackage{ifxetex,ifluatex}
\usepackage{fixltx2e} % provides \textsubscript
\ifnum 0\ifxetex 1\fi\ifluatex 1\fi=0 % if pdftex
  \usepackage[T1]{fontenc}
  \usepackage[utf8]{inputenc}
\else % if luatex or xelatex
  \ifxetex
    \usepackage{mathspec}
    \usepackage{xltxtra,xunicode}
  \else
    \usepackage{fontspec}
  \fi
  \defaultfontfeatures{Mapping=tex-text,Scale=MatchLowercase}
  \newcommand{\euro}{€}
\fi
% use upquote if available, for straight quotes in verbatim environments
\IfFileExists{upquote.sty}{\usepackage{upquote}}{}
% use microtype if available
\IfFileExists{microtype.sty}{%
\usepackage{microtype}
\UseMicrotypeSet[protrusion]{basicmath} % disable protrusion for tt fonts
}{}
\usepackage[margin=1in]{geometry}
\ifxetex
  \usepackage{polyglossia}
  \setmainlanguage{}
\else
  \usepackage[shorthands=off,french]{babel}
\fi
\usepackage{color}
\usepackage{fancyvrb}
\newcommand{\VerbBar}{|}
\newcommand{\VERB}{\Verb[commandchars=\\\{\}]}
\DefineVerbatimEnvironment{Highlighting}{Verbatim}{commandchars=\\\{\}}
% Add ',fontsize=\small' for more characters per line
\usepackage{framed}
\definecolor{shadecolor}{RGB}{248,248,248}
\newenvironment{Shaded}{\begin{snugshade}}{\end{snugshade}}
\newcommand{\AlertTok}[1]{\textcolor[rgb]{0.94,0.16,0.16}{#1}}
\newcommand{\AnnotationTok}[1]{\textcolor[rgb]{0.56,0.35,0.01}{\textbf{\textit{#1}}}}
\newcommand{\AttributeTok}[1]{\textcolor[rgb]{0.77,0.63,0.00}{#1}}
\newcommand{\BaseNTok}[1]{\textcolor[rgb]{0.00,0.00,0.81}{#1}}
\newcommand{\BuiltInTok}[1]{#1}
\newcommand{\CharTok}[1]{\textcolor[rgb]{0.31,0.60,0.02}{#1}}
\newcommand{\CommentTok}[1]{\textcolor[rgb]{0.56,0.35,0.01}{\textit{#1}}}
\newcommand{\CommentVarTok}[1]{\textcolor[rgb]{0.56,0.35,0.01}{\textbf{\textit{#1}}}}
\newcommand{\ConstantTok}[1]{\textcolor[rgb]{0.00,0.00,0.00}{#1}}
\newcommand{\ControlFlowTok}[1]{\textcolor[rgb]{0.13,0.29,0.53}{\textbf{#1}}}
\newcommand{\DataTypeTok}[1]{\textcolor[rgb]{0.13,0.29,0.53}{#1}}
\newcommand{\DecValTok}[1]{\textcolor[rgb]{0.00,0.00,0.81}{#1}}
\newcommand{\DocumentationTok}[1]{\textcolor[rgb]{0.56,0.35,0.01}{\textbf{\textit{#1}}}}
\newcommand{\ErrorTok}[1]{\textcolor[rgb]{0.64,0.00,0.00}{\textbf{#1}}}
\newcommand{\ExtensionTok}[1]{#1}
\newcommand{\FloatTok}[1]{\textcolor[rgb]{0.00,0.00,0.81}{#1}}
\newcommand{\FunctionTok}[1]{\textcolor[rgb]{0.00,0.00,0.00}{#1}}
\newcommand{\ImportTok}[1]{#1}
\newcommand{\InformationTok}[1]{\textcolor[rgb]{0.56,0.35,0.01}{\textbf{\textit{#1}}}}
\newcommand{\KeywordTok}[1]{\textcolor[rgb]{0.13,0.29,0.53}{\textbf{#1}}}
\newcommand{\NormalTok}[1]{#1}
\newcommand{\OperatorTok}[1]{\textcolor[rgb]{0.81,0.36,0.00}{\textbf{#1}}}
\newcommand{\OtherTok}[1]{\textcolor[rgb]{0.56,0.35,0.01}{#1}}
\newcommand{\PreprocessorTok}[1]{\textcolor[rgb]{0.56,0.35,0.01}{\textit{#1}}}
\newcommand{\RegionMarkerTok}[1]{#1}
\newcommand{\SpecialCharTok}[1]{\textcolor[rgb]{0.00,0.00,0.00}{#1}}
\newcommand{\SpecialStringTok}[1]{\textcolor[rgb]{0.31,0.60,0.02}{#1}}
\newcommand{\StringTok}[1]{\textcolor[rgb]{0.31,0.60,0.02}{#1}}
\newcommand{\VariableTok}[1]{\textcolor[rgb]{0.00,0.00,0.00}{#1}}
\newcommand{\VerbatimStringTok}[1]{\textcolor[rgb]{0.31,0.60,0.02}{#1}}
\newcommand{\WarningTok}[1]{\textcolor[rgb]{0.56,0.35,0.01}{\textbf{\textit{#1}}}}
\ifxetex
  \usepackage[setpagesize=false, % page size defined by xetex
              unicode=false, % unicode breaks when used with xetex
              xetex]{hyperref}
\else
  \usepackage[unicode=true]{hyperref}
\fi
\hypersetup{breaklinks=true,
            bookmarks=true,
            pdfauthor={Kim Antunez},
            pdftitle={Commentaires sur les travaux d'Alain et Romain},
            colorlinks=true,
            citecolor=\#FF0000,
            urlcolor=\#FF0000,
            linkcolor=\#FF0000,
            pdfborder={0 0 0}}
\urlstyle{same}  % don't use monospace font for urls
\setlength{\parindent}{0pt}
\setlength{\parskip}{6pt plus 2pt minus 1pt}
\setlength{\emergencystretch}{3em}  % prevent overfull lines
\setcounter{secnumdepth}{5}

%%% Use protect on footnotes to avoid problems with footnotes in titles
\let\rmarkdownfootnote\footnote%
\def\footnote{\protect\rmarkdownfootnote}


  \title{Commentaires sur les travaux d'Alain et Romain}
    \author{Kim Antunez}
    \date{}
  

\begin{document}

\maketitle


{
\hypersetup{linkcolor=black}
\setcounter{tocdepth}{2}
\tableofcontents
}
\hypertarget{cruxe9er-un-rapport-automatique}{%
\section{Créer un rapport
automatique}\label{cruxe9er-un-rapport-automatique}}

\hypertarget{rendu-de-rmarkdown-avec-du-code-python}{%
\subsection{Rendu de Rmarkdown avec du code
python}\label{rendu-de-rmarkdown-avec-du-code-python}}

Exemple à partir des premières lignes du code d'Alain.

\begin{Shaded}
\begin{Highlighting}[]
\ImportTok{import}\NormalTok{ os}
\ImportTok{import}\NormalTok{ string}
\ImportTok{import}\NormalTok{ re}
\ImportTok{import}\NormalTok{ math}
\ImportTok{from}\NormalTok{ math }\ImportTok{import}\NormalTok{ sqrt}
\ImportTok{import}\NormalTok{ numpy }\ImportTok{as}\NormalTok{ np}
\ImportTok{import}\NormalTok{ random}
\ImportTok{import}\NormalTok{ time}
\NormalTok{random.seed(}\DecValTok{1}\NormalTok{)}


\NormalTok{os.chdir(}\StringTok{'C:/Users/Kim Antunez/Documents/Projets_autres'}\NormalTok{)}
\BuiltInTok{print}\NormalTok{(string.punctuation }\OperatorTok{+} \StringTok{"'’"}\NormalTok{)}
\end{Highlighting}
\end{Shaded}

\begin{verbatim}
## !"#$%&'()*+,-./:;<=>?@[\]^_`{|}~'’
\end{verbatim}

\begin{Shaded}
\begin{Highlighting}[]
\KeywordTok{def}\NormalTok{ mise_en_forme_phrase (phrase):}
\NormalTok{    phrase }\OperatorTok{=}\NormalTok{ phrase.lower()}
    \CommentTok{# On elève la ponctuation mais ça peut se discuter (garder les @ et #?)}
\NormalTok{    phrase }\OperatorTok{=}\NormalTok{ re.sub(}\StringTok{'( @[^ ]*)|(^@[^ ]*)'}\NormalTok{,}\StringTok{"nickname"}\NormalTok{, phrase) }\CommentTok{#Remplace @... par nickname}
    \CommentTok{#supprime toutes les ponctuations par défaut + les apostrophes bizarres}
\NormalTok{    phrase }\OperatorTok{=}\NormalTok{ phrase.translate(}\BuiltInTok{str}\NormalTok{.maketrans(}\StringTok{''}\NormalTok{, }\StringTok{''}\NormalTok{, string.punctuation }\OperatorTok{+} \StringTok{"'’"}\NormalTok{))}
    \CommentTok{# On enlève les passages à la ligne}
\NormalTok{    phrase }\OperatorTok{=}\NormalTok{ re.sub(}\StringTok{'}\CharTok{\textbackslash{}\textbackslash{}}\StringTok{n'}\NormalTok{, }\StringTok{' '}\NormalTok{, phrase)}
    \CommentTok{# On enlève les espaces multiples et les espaces à la fin des phrases}
\NormalTok{    phrase }\OperatorTok{=}\NormalTok{ re.sub(}\StringTok{' +'}\NormalTok{, }\StringTok{' '}\NormalTok{, phrase)}
\NormalTok{    phrase }\OperatorTok{=}\NormalTok{ re.sub(}\StringTok{' +$'}\NormalTok{, }\StringTok{''}\NormalTok{, phrase)}
    \ControlFlowTok{return}\NormalTok{(phrase.split())}
\CommentTok{#f = open('data/sample_3.txt')}
\CommentTok{#raw = f.read()}
\CommentTok{#print(type(raw))}
\ControlFlowTok{with} \BuiltInTok{open}\NormalTok{(}\StringTok{'data/sample_3.txt'}\NormalTok{, encoding}\OperatorTok{=}\StringTok{"utf-8"}\NormalTok{) }\ImportTok{as}\NormalTok{ myfile:}
\NormalTok{    phrases }\OperatorTok{=}\NormalTok{ [mise_en_forme_phrase(}\BuiltInTok{next}\NormalTok{(myfile)) }\ControlFlowTok{for}\NormalTok{ x }\KeywordTok{in} \BuiltInTok{range}\NormalTok{(}\DecValTok{10000}\NormalTok{)]}
\BuiltInTok{print}\NormalTok{(phrases[}\DecValTok{0}\NormalTok{:}\DecValTok{1}\NormalTok{])}
\CommentTok{#raw = ''.join([''.join(phrase) for phrase in phrases])}
\end{Highlighting}
\end{Shaded}

\begin{verbatim}
## [['il', 'mérite', 'd', 'être', 'bloquer', 'la', 'lettre', 'de', 'l', 'alphabet']]
\end{verbatim}

\hypertarget{commentaires-des-codes-dar}{%
\section{Commentaires des codes
d'A\&R}\label{commentaires-des-codes-dar}}

\hypertarget{alain}{%
\subsection{Alain}\label{alain}}

\hypertarget{autograd-est-depreciated}{%
\subsubsection{autograd est
depreciated}\label{autograd-est-depreciated}}

L'option suivante semble depreciated :

\begin{Shaded}
\begin{Highlighting}[]
\BuiltInTok{input} \OperatorTok{=}\NormalTok{ autograd.Variable(}\BuiltInTok{input}\NormalTok{, requires_grad}\OperatorTok{=}\VariableTok{True}\NormalTok{)}
\CommentTok{#UserWarning: torch.autograd.variable(...) is deprecated, use torch.tensor(...) instead}
\CommentTok{#UserWarning: To copy construct from a tensor, it is recommended to use sourceTensor.clone().detach() or  sourceTensor.clone().detach().requires_grad_(True), rather than torch.tensor(sourceTensor)}
\end{Highlighting}
\end{Shaded}

J'ai remplacé par ça et le warning a disparu :

\begin{Shaded}
\begin{Highlighting}[]
\BuiltInTok{input} \OperatorTok{=}\NormalTok{ Variable(torch.Tensor(}\BuiltInTok{input}\NormalTok{), requires_grad}\OperatorTok{=}\VariableTok{True}\NormalTok{) }\CommentTok{#KIM}
\end{Highlighting}
\end{Shaded}

Ou sûrement mieux en une étape : cf.~ce qu'a fait Romain :

\begin{Shaded}
\begin{Highlighting}[]
\NormalTok{W1 }\OperatorTok{=}\NormalTok{ Variable(torch.randn(embedding_dims, voc_size).}\BuiltInTok{float}\NormalTok{(), requires_grad}\OperatorTok{=}\VariableTok{True}\NormalTok{)}
\end{Highlighting}
\end{Shaded}

Faire la même chose avec ``output''

\hypertarget{romain}{%
\subsection{Romain}\label{romain}}

RAS à première vue.

\hypertarget{croisement-des-deux-codes}{%
\subsection{Croisement des deux codes}\label{croisement-des-deux-codes}}

\hypertarget{w2-faut-il-la-transposer}{%
\subsubsection{W2, faut-il la transposer
?}\label{w2-faut-il-la-transposer}}

Pour W2 ou output vous avez la même chose à une transposée près :
choisir ce qui est le mieux entre les deux pour cohérence avec
explication du modèle

\hypertarget{etape-de-cruxe9ation-des-couples-contexte-target}{%
\subsubsection{Etape de création des couples contexte /
target}\label{etape-de-cruxe9ation-des-couples-contexte-target}}

Différence d'approche pour la création des couples contexte target.
Alors qu'Alain définit les couples dans une fonctions à part, Romain le
fait à l'intérieur de l'algorithme de mise à jour de W1 et W2.
Intuitivement je dirais que les 2 approches sont équivalentes.

\hypertarget{deux-approches-diffuxe9rentes-pour-le-produit-matriciel}{%
\subsubsection{Deux approches différentes pour le produit
matriciel}\label{deux-approches-diffuxe9rentes-pour-le-produit-matriciel}}

Alain le fait en une étape mais avec une fonction plus etoffée de
création des targets contexte (creer\_echantillon)

\begin{Shaded}
\begin{Highlighting}[]
\NormalTok{data }\OperatorTok{=}\NormalTok{ torch.matmul(}\BuiltInTok{input}\NormalTok{[focus,], torch.t(output))}
\end{Highlighting}
\end{Shaded}

Romain décompose comme cela

\begin{Shaded}
\begin{Highlighting}[]
\NormalTok{x }\OperatorTok{=}\NormalTok{ Variable(get_input_layer(focus)).}\BuiltInTok{float}\NormalTok{()}
\NormalTok{y }\OperatorTok{=}\NormalTok{ Variable(torch.from_numpy(np.array([context])).}\BuiltInTok{long}\NormalTok{())}
\NormalTok{z1 }\OperatorTok{=}\NormalTok{ torch.matmul(W1, x)}
\NormalTok{z2 }\OperatorTok{=}\NormalTok{ torch.matmul(W2, z1)}
\end{Highlighting}
\end{Shaded}

Idem je sais pas ce qui est mieux.

\hypertarget{une-fonction-de-loss-en-une-ligne-ou-deux}{%
\subsubsection{Une fonction de loss en une ligne ou deux
?}\label{une-fonction-de-loss-en-une-ligne-ou-deux}}

Déjà évoqué alors que Romain utilise cela

\begin{Shaded}
\begin{Highlighting}[]
\NormalTok{log_softmax }\OperatorTok{=}\NormalTok{ F.log_softmax(z2, dim}\OperatorTok{=}\DecValTok{0}\NormalTok{)}
\NormalTok{loss }\OperatorTok{=}\NormalTok{ F.nll_loss(log_softmax.view(}\DecValTok{1}\NormalTok{,}\OperatorTok{-}\DecValTok{1}\NormalTok{), y)}
\end{Highlighting}
\end{Shaded}

Alain le fait en une étape

\begin{Shaded}
\begin{Highlighting}[]
\NormalTok{loss }\OperatorTok{=}\NormalTok{ F.cross_entropy(data.view(}\DecValTok{1}\NormalTok{,}\OperatorTok{-}\DecValTok{1}\NormalTok{), torch.tensor([context]))}
\end{Highlighting}
\end{Shaded}

\hypertarget{todo-list-pour-fuxe9vrier}{%
\section{TODO-LIST pour février}\label{todo-list-pour-fuxe9vrier}}

\begin{itemize}
\item
  Kim : comprendre le modèle suite à cours particulier d'Alain et Romain
  puis replonger simplement dans la décomposition des étapes du modèle
  (multiplication des matrices, descente de gradients)
\item
  Tous : faire UN modèle unique commun à tous. Pour cela créer des
  fonctions avec des options du type ``negative\_sampling = TRUE /
  FALSE'' pour éviter la multiplication des fichiers. Voir intégrer le
  modèle dans un ``package'' pour ensuite avoir un fichier ou on le fait
  tourner sur les données tests et un fichier ou on le fait tourner sur
  les vrais tweets. De même créer des fonctions pour ``évaluer'' les
  sorties du modèle (fonctions ACP, ACP\_interactive, TSNE,
  TSNE\_interactive)
\item
  Tous : une fois ces fonctions créées, estimer les étapes qui durent
  longtemps (avec des fonctionnalités python) et les optimiser si
  possible\ldots{}
\item
  Tous : faire tourner modèle + évaluation sur l'ensemble des tweets à
  notre dispo et voir combien de temps ça met
\item
  Tous : implémenter l'évaluation avec nearest neighbor et human
  judgment agreement.
\end{itemize}

\end{document}